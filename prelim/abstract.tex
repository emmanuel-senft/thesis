\textbf{TEACHING ROBOTS SOCIAL AUTONOMY FROM IN SITU HUMAN SUPERVISION}\newline
\textbf{Emmanuel Senft}

Traditionally the behaviour of social robots has been programmed. However, increasingly there has been a focus on letting robots learn their behaviour to some extent from example or through trial and error. This on the one hand excludes the need for programming, but also allows the robot to adapt to circumstances not foreseen at the time of programming. One such occasion is when the user wants to tailor or fully specify the robot’s behaviour. The engineer often has limited knowledge of what the user wants or what the deployment circumstances specifically require. Instead, the user does know what is expected from the robot and consequently, the social robot should be equipped with a mechanism to learn from its user.

This work explores how a social robot can learn to interact meaningfully with people in an efficient and safe way by learning from supervision by a human teacher in control of the robot's behaviour. To this end we propose a new machine learning framework called \gls{sparc}. \gls{sparc} enables non-technical users to control and teach a robot, and we evaluate its effectiveness in \gls{hri}. The core idea is that the user initially remotely operates the robot, while an algorithm associates actions to states and gradually learns. Over time, the robot takes over the control from the user while still giving the user oversight of the robot’s behaviour by ensuring that every action executed by the robot has been actively or passively approved by the user. This is particularly important in \gls{hri}, as interacting with people, and especially vulnerable users, is a complex and multidimensional problem, and any errors by the robot may have negative consequences for the people involved in the interaction. 

Through the development and evaluation of \gls{sparc}, this work contributes to both \gls{hri} and \acrlong{iml}, especially on how autonomous agents, such as social robots, can learn from people and how this specific teacher-robot interaction impacts the learning process. We showed that a supervised robot learning from their user can reduce the workload of this person, and that providing the user with the opportunity to control the robot's behaviour substantially improves the teaching process. Finally, this work also demonstrated that a robot supervised by a user could learn rich social behaviours in the real world, in a large multidimensional and multimodal sensitive environment, as a robot learned quickly (25 interactions of 4 sessions during in average 1.9 minutes) to tutor children in an educational game, achieving similar behaviours and educational outcomes compared to a robot fully controlled by the user, both providing 10 to 30\% improvement in game metrics compared to a passive robot. 