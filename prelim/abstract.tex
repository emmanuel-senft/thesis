\textbf{TEACHING ROBOTS SOCIAL AUTONOMY FROM IN-SITU HUMAN SUPERVISION}\newline
\textbf{Emmanuel Senft}

To reach the potential promised to robot in interaction with humans, robots need to learn from humans. Knowledge of expected robot behaviours does not lie in engineers' hand but in the end-users. As a consequence, to reach new applications and enter daily lives, end-users need to be provided with a way to teach their robot to interact in the way they desire.

The thesis explored by this work is that a robot can learn to interact meaningfully with humans in an efficient and safe way by receiving supervision from a human teacher in control of the robot's behaviour. This work original contribution to knowledge is a new teaching framework applicable to robotics and addressing the thesis: \gls{sparc} and its evaluation in three studies. \gls{sparc} aims at enabling non-technical users to teach a robot in high stakes environments such as \gls{hri}. \gls{sparc} provides the teacher with control over the robot's action to ensure that every action executed by the robot has been actively or passively accepted by the teacher. This framework has been applied to teach a robot to tutor children in an educational game, and the autonomous achieved similar results to a supervised robot.

Throughout evaluating \gls{sparc}, this work contributes to extend our knowledge on how human teach autonomous agents and the impacts of features of the teacher-robot interface on the teaching process. It is found here that a supervised robot learning from a human could reduce the workload required to express a useful robot behaviour. Furthermore, providing the teacher with the opportunity to control the robot's behaviour improves substantially the teaching process.




%This work original contribution to knowledge is a new teaching paradigm for robotics: \gls{sparc} which has been validated in three studies: interaction with a model, interaction with a fixed environment and interaction with humans in the context of robotic tutors.

%First? demonstration of teaching a robot online to interact with humans