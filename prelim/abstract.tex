\textbf{TEACHING ROBOTS SOCIAL AUTONOMY FROM IN-SITU HUMAN SUPERVISION}\newline
\textbf{Emmanuel Senft}

Traditionally the behaviour of social robots has been programmed. However, increasingly there has been a focus on letting robots learn their behaviour to some extent from example or through trial and error. This on the one hand excludes the need for programming, but also allows the robot to adapt to circumstances not foreseen at the time of programming. One such occasion is when the user wants to tailor or fully specify the robot’s behaviour. The engineer often has limited knowledge of what the user wants or what the deployment circumstances specifically require. Instead, the user does know what is expected from the robot and consequently, the social robot should be equipped with a mechanism to learn from its user.
%To reach the potential promised to robots in interaction with humans, robots need to learn from them. Knowledge of expected robot behaviours does not lie in engineers' hand but in the end-users'. As a consequence, to reach new applications and enter daily lives, robot users need to be provided with a way to teach their robot to interact in the way they desire.

This work explores how a social robot can learn to interact meaningfully with people in an efficient and safe way by learning from supervision by a human teacher in control of the robot's behaviour. To this end we propose a new machine learning framework called \gls{sparc}. \gls{sparc} enables non-technical users to control and teach a robot, and we evaluate its effectiveness in \gls{hri}. The core idea is that the user initially remotely operates the robot, while an algorithm associates actions to states and gradually learns. Over time, the robot takes over the control from the user while still giving the user oversight of the robot’s behaviour by ensuring that every action executed by the robot has been actively or passively approved by the user. This is particularly important in \gls{hri}, as interacting with people, and especially vulnerable users, is a complex and multidimensional problem, and any errors by the robot may have negative consequences for the people involved in the interaction. 
%The thesis explored by this work is that a robot can learn to interact meaningfully with humans in an efficient and safe way by receiving supervision from a human teacher in control of the robot's behaviour. This work's original contribution to knowledge is a new teaching framework applicable to robotics and addressing the thesis: the \gls{sparc} and its evaluation in three studies. \gls{sparc} aims at enabling non-technical users to teach a robot in high-stakes environments such as \gls{hri}. By providing the teacher with control over the robot, \gls{sparc} ensures that every action executed by the robot has been actively or passively approved by the teacher. Learning to interact with humans is a challenge as the environment is complex, multidimensional and social, and errors from the robot may have dramatic impacts on the humans involved in the interaction. Nevertheless, \gls{sparc} demonstrates its applicability to \gls{hri} by successfully teaching a robot to tutor children in an educational game, resulting similar behaviours for the autonomous robot and one controlled by a human. 

Through the development and evaluation of \gls{sparc}, this work contributes to both \gls{hri} and \acrlong{iml}, especially on how autonomous agents, such as social robots, can learn from people and what the impact is of the human-robot interface on the learning process. We showed that a supervised robot learning from their user can reduce the workload of this person, and that providing the user with the opportunity to control the robot's behaviour substantially improves the teaching process. Finally, this work also demonstrated that a robot supervised by a user could learn rich social behaviours in a large multidimensional and multimodal high-stakes environment, as a robot learned how to tutor children in an educational game, achieving similar behaviours and educational outcomes compared to a robot fully controlled by the user. 
%Through evaluating \gls{sparc}, this work contributes to extend our knowledge both in \gls{hri} and \acrlong{iml}, especially on how humans teach autonomous agents and the impacts of the teacher-robot interface's features on the teaching process. It is found here that a supervised robot learning from a human could reduce the workload required to express a useful robot behaviour and that providing the teacher with the opportunity to control the robot's behaviour improves substantially the teaching process. Finally, this work also demonstrated that a robot supervised by a human could learn a complex social behaviour in a large multidimensional and multimodal high-stakes environment.

%Sev
%To achieve the potential promised to robots interacting with humans, robots must to learn from them. The knowledge of the behaviors expected from robots does not lie in the hand but at the end users. As a result, to reach new applications and enter daily life, robot users need a way to teach their robot to interact in as they wish.
%The thesis explored by this work is that a robot can learn to interact significantly with in an efficient and safe manner by receiving the supervision of a human teacher control of the behavior of the robot. This work original contribution to knowledge is a new educational framework applicable to robotics and addressing the thesis: the SPARC (Progressively Autonomous Robot Skills) and its evaluation in three studies. SPARC aims to allow non-technical users to teach a robot in high stakes environments such as Human-Robot Interaction (HRI). By providing the teacher control on the robot, SPARC ensures that every action performed by the robot has been actively or passively approved by the teacher. This approach has demonstrated Applicability to HRI by successfully teaching a robot for tutoring children in an educational environment play, resulting from similar behaviors for the autonomous robot and the one controlled by a human.
%Throughout the SPARC evaluation, this work contributes to expanding our knowledge of humans  each autonomous agents and the impacts of the teacher-robot interface features on the teaching process. Here is a supervised learning robot of a human could reduce the workload needed to express a useful robot behavior. In addition, provide the teacher with the opportunity to control the behavior of the robot significantly improves the teaching process and can lead to an autonomous robot interact effectively with humans.

%This work original contribution to knowledge is a new teaching paradigm for robotics: \gls{sparc} which has been validated in three studies: interaction with a model, interaction with a fixed environment and interaction with humans in the context of robotic tutors.

%First? demonstration of teaching a robot online to interact with humans