\textbf{TEACHING ROBOTS SOCIAL AUTONOMY FROM IN-SITU HUMAN SUPERVISION}\newline
\textbf{Emmanuel Senft}

To reach the potential promised to robots in interaction with humans, robots need to learn from them. Knowledge of expected robot behaviours does not lie in engineers' hand but in the end-users'. As a consequence, to reach new applications and enter daily lives, robot users need to be provided with a way to teach their robot to interact in the way they desire.

The thesis explored by this work is that a robot can learn to interact meaningfully with humans in an efficient and safe way by receiving supervision from a human teacher in control of the robot's behaviour. This work original contribution to knowledge is a new teaching framework applicable to robotics and addressing the thesis: the \gls{sparc} and its evaluation in three studies. \gls{sparc} aims at enabling non-technical users to teach a robot in high-stakes environments such as \gls{hri}. By providing the teacher with control over the robot, \gls{sparc} ensures that every action executed by the robot has been actively or passively approved by the teacher. This approach demonstrated its applicability to \gls{hri} by successfully teaching a robot to tutor children in an educational game, resulting similar behaviours for the autonomous robot and the one controlled by a human. 

Throughout evaluating \gls{sparc}, this work contributes to extend our knowledge on how humans teach autonomous agents and the impacts of the teacher-robot interface's features on the teaching process. It is found here that a supervised robot learning from a human could reduce the workload required to express a useful robot behaviour and that providing the teacher with the opportunity to control the robot's behaviour improves substantially the teaching process. Finally, we also showed that a robot supervised by a human could learn a complex social behaviour in a large multidimensional and multimodal high-stakes environments

%Sev
%To achieve the potential promised to robots interacting with humans, robots must to learn from them. The knowledge of the behaviors expected from robots does not lie in the hand but at the end users. As a result, to reach new applications and enter daily life, robot users need a way to teach their robot to interact in as they wish.
%The thesis explored by this work is that a robot can learn to interact significantly with in an efficient and safe manner by receiving the supervision of a human teacher control of the behavior of the robot. This work original contribution to knowledge is a new educational framework applicable to robotics and addressing the thesis: the SPARC (Progressively Autonomous Robot Skills) and its evaluation in three studies. SPARC aims to allow non-technical users to teach a robot in high stakes environments such as Human-Robot Interaction (HRI). By providing the teacher control on the robot, SPARC ensures that every action performed by the robot has been actively or passively approved by the teacher. This approach has demonstrated Applicability to HRI by successfully teaching a robot for tutoring children in an educational environment play, resulting from similar behaviors for the autonomous robot and the one controlled by a human.
%Throughout the SPARC evaluation, this work contributes to expanding our knowledge of humans  each autonomous agents and the impacts of the teacher-robot interface features on the teaching process. Here is a supervised learning robot of a human could reduce the workload needed to express a useful robot behavior. In addition, provide the teacher with the opportunity to control the behavior of the robot significantly improves the teaching process and can lead to an autonomous robot interact effectively with humans.

%This work original contribution to knowledge is a new teaching paradigm for robotics: \gls{sparc} which has been validated in three studies: interaction with a model, interaction with a fixed environment and interaction with humans in the context of robotic tutors.

%First? demonstration of teaching a robot online to interact with humans