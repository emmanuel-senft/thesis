%%%%%%%%%%%%%%%%%%%%%%%%%%%%%%%%%%%%%%%%%%%%%%%%%%%%%
\chapter{Introduction} \label{chap:intro}

HRI what it is and what it matters and why it is challenging

Robots will inhabit human spaces and need to interact with them in decent ways

%%%%%%%%%%%%%%%%%%%%%%%%%%%%%%%%%%%%%%%%%%%%%%%%%%%%%
\section{Scope}\label{sec:intro-scope}

\subsection{Frame}

\subsection{Environment} \label{sec:scope-social}
Robot interacting in an environment shared with humans / directly with humans
presence of a supervisor who can provide feedback/commands

\subsection{Type of interaction}


\subsection{Algorithms}


%%%%%%%%%%%%%%%%%%%%%%%%%%%%%%%%%%%%%%%%%%%%%%%%%%%%%
\section{The Thesis}\label{sec:intro-thesis}
The main thesis that this document seeks to put forward is as below.

One sentence to describe the take home/conclusion

research questions explored

%%%%%%%%%%%%%%%%%%%%%%%%%%%%%%%%%%%%%%%%%%%%%%%%%%%%%
\section{Approach and Experimentation}\label{sec:intro-exps}

%%%%%%%%%%%%%%%%%%%%%%%%%%%%%%%%%%%%%%%%%%%%%%%%%%%%%
\section{Key Concepts}\label{sec:intro-concepts}

%%%%%%%%%%%%%%%%%%%%%%%%%%%%%%%%%%%%%%%%%%%%%%%%%%%%%
\section{Challenges}

%%%%%%%%%%%%%%%%%%%%%%%%%%%%%%%%%%%%%%%%%%%%%%%%%%%%%
\section{Contributions}\label{sec:intro-contr}

\begin{itemize}
	\item Something 
\end{itemize}

%%%%%%%%%%%%%%%%%%%%%%%%%%%%%%%%%%%%%%%%%%%%%%%%%%%%%
\section{Structure}\label{sec:intro-struct}
The structure of this thesis is outlined below to provide an overview of the content and context for each chapter. A summary of key experimental findings are included at the start of each relevant chapter for ease of reference.

\begin{itemize}
\item This chapter provided an introduction to the general field of this research (robot tutors for children), the research questions including the central \textit{thesis}, scope, and contributions of the work presented in later chapters.

\item Chapter \ref{chap:background} 

\item Chapter \ref{chap:method}

\item Chapter \ref{chap:maindisc} draws on the experimental work from previous chapters, alongside the context supplied by related work, to form a discussion about the broader context and findings of the thesis. Limitations of the work conducted here are outlined, leading to suggestions for future directions of research.

\item Chapter \ref{chap:conclusion} concludes the thesis with a summary of the main contributions.

\end{itemize}
