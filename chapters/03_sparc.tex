\chapter{Supervised Progressive Autonomous Robot Competencies}\label{chap:sparc}

\begin{framed}
	\textbf{Key points:}
	
	\begin{itemize}
		\item Novel interaction framework to teach robots how to interact while interacting
		\item Provide the human teacher control on the robot actions and learns from the supervision
		\item The robot behaviour (under supervision) can be assumed to be optimal
		\item Reduce workload on the teacher over time as the robot learns
	\end{itemize}
\end{framed}

Parts of the work presented in this chapter have been published in \cite{senft2015sparc}. The final publication is available from Springer via \url{http://dx.doi.org/10.1007/978-3-319-25554-5_60}.

\newpage

\section{Principle}

\cite{senft2015sparc}

Keeping a human teacher in the loop and in control of the robot's actions,
equivalent to level 6 on the Sheridan scale of autonomy: "A computer selects
action, informs human in plenty of time to stop it" \cite{sheridan1978human}.

Learning in the background aims to improve the correctness of suggested actions
to decrease the chance of being corrected.

Human can provide additional information to the algorithm to speed up the
learning

\section{Goal}

Reach quickly an appropriate action policy using teacher's demonstration,
feedback and highligh of features

Reduce workload over time as the suggestions become accurate

Maintain a correct action policy throughout the learning

\section{Frame}

Requires a human supervisor: dyadic interaction (Supervisor-Robot-Supervisor or
Supervisor-Robot-Environment) or triadic interaction (Supervisor-Robot-Human
partner)

Timescale (Limite to human rate of action selection/correction - can be dealt
with using higher level actions or goals)

\section{Interaction with Machine Learning Algorithms}

Supervised Learning

Reinforcement learning 

Other type of learning (ex: instance based learning)

\section{Summary}
    
    

