\chapter{Contribution and Conclusion} \label{chap:conclusion}
%rephrase
This chapter seeks to provide an overview of the findings and topics covered in this thesis. The contributions to the field of social \gls{hri} are outlined and summarised. Following this, a conclusion is provided to briefly encapsulate the primary outcome of this work.

%%%%%%%%%%%%%%%%%%%%%%%%%%%%%%%%%%%%%%%%%%%%%%%%%%%%%%%%%%%%%%%%%%%%%%%%%%
\section{Summary}\label{sec:conc_summary}
%%%%%%%%%%%%%%%%%%%%%%%%%%%%%%%%%%%%%%%%%%%%%%%%%%%%%%%%%%%%%%%%%%%%%%%%%%

%Difference contributions/summary??
\section{Contributions}\label{sec:conc_contribution}
%rephrase
This section will revisit the contributions outlined in the introduction (Chapter \ref{chap:intro}), with further expansion and explanation. 
%\subsection{Scientific Contributions}
The main contributions of this thesis are as follows:
\begin{itemize}
	\item \textbf{Design of a new interaction framework for teaching agents in a safe way.} One of the main contribution of this research is \gls{sparc}: an interaction framework enabling robot to learn to interact with humans from in situ human supervision. This method is corner stone of this thesis and aims to allow robots to be adaptive, while constantly possessing an appropriate action policy and require a low workload from humans.
	
	\item \textbf{Evaluation of \gls{sparc} in three studies.} Throughout this research, \gls{sparc} have been tested in three study, which were developed to explores different human-robot interactions involved in \gls{sparc}. First study involved teaching a robot to interact with another robot, simulating a 
	
	\item \textbf{Demonstration of the importance of control over the robot's action when teaching a robot to interact.} S
	
	\item \textbf{Design of a lightweight algorithm to quickly learn from demonstration in complex environments.}
	
	\item \textbf{Application of \gls{iml} to safely teach robots social autonomy from in situ human supervision.}
\end{itemize}

%\subsection{Technical Contributions}
%In addition to the scientific contributions, this research project lead to software development for multiple projects.
%\begin{itemize}
%	\item Development of three different studies to evaluate \gls{sparc} in three different scenarios (supervised robot-robot interaction, virtual robot and real word \gls{hri}).
%	\item Partial development of a cognitive architecture and two tools for the \acrshort{dream} project (European FP7 project: 611391).
%	\item Development of an autonomous robot controller to support cardiac rehabilitation in the Human-Robot Interaction Strategies for Rehabilitation based on Socially Assistive Robotics project (Royal Academy of Engineering: IAPP\textbackslash1516\textbackslash137).
%	\item Development of a wizard interface for the Freeplay-Sandbox\footnote{\url{https://github.com/freeplay-sandbox}}. 
%\end{itemize}
%%%%%%%%%%%%%%%%%%%%%%%%%%%%%%%%%%%%%%%%%%%%%%%%%%%%%%%%%%%%%%%%%%%%%%%%%%
\section{Conclusion}\label{sec:conc_conc}
