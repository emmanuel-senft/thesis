\chapter{Contribution and Conclusion} \label{chap:conclusion}
\glsresetall
%rephrase
\ES{From James}
This chapter seeks to provide an overview of the findings and topics covered in this thesis. The contributions to the field of social \gls{hri} are outlined and summarised. Following this, a conclusion is provided to briefly encapsulate the primary outcome of this work.

%%%%%%%%%%%%%%%%%%%%%%%%%%%%%%%%%%%%%%%%%%%%%%%%%%%%%%%%%%%%%%%%%%%%%%%%%%
\section{Summary}\label{sec:conc_summary}
%%%%%%%%%%%%%%%%%%%%%%%%%%%%%%%%%%%%%%%%%%%%%%%%%%%%%%%%%%%%%%%%%%%%%%%%%%

The main thesis presented by this work is the following one: 
\begin{quote}
	A robot can learn to interact meaningfully with humans in an efficient and safe way by receiving supervision from a human teacher in control of the robot's behaviour. 
\end{quote}

To explore this thesis and the research questions arising when addressing it, we first reviewed the field of application of social \gls{hri}. From this overview, we drew three requirements for a robot to interact efficiently in human-environments. The robot needs to constantly have an appropriate action policy, be adaptive and require a low workload from humans. Then we analysed the current controller for robots in \gls{hri} and identified the absence of a robot controller validating these requirements (cf. Chapter \ref{chap:background}).

In Chapter \ref{chap:sparc} we aimed to address this lack of method validating this requirements by proposing a new approach, the \gls{sparc}, to teach robots to interact and which would be applicable to \gls{hri}. \gls{sparc} aims to provide control over the robot's action to a human and use this supervision to learn online a correct action policy. To achieve this goal, \gls{sparc} is based on a set of principles allowing this teaching interaction to be efficient: the teacher can select actions for the robot, the robot can propose actions to the teacher, the teacher can pre-empt robot's propositions before an automatic execution and, finally, the robot learns from the teacher's selections and feedback to improve its action policy.

Once the method and the principles defining the interaction between the human teacher and the robot were set, we evaluated this approach in three studies with increasing ambition. Before testing \gls{sparc} in a real world interaction with humans (in Chapter \ref{chap:tutoring}), we needed to evaluate if the principles underlying \gls{sparc} could reduce the workload on the teacher and how the control over the robot's action impacts the teaching process, especially, is it sufficient to ensure a constant appropriate robot behaviour. The two first studies (Chapters \ref{chap:woz} and \ref{chap:control}) included humans only as teachers, but not as the target of the teaching. This allowed to gather initial information on \gls{sparc} in controlled and repeatable environments without having to tackle the challenges of interacting with humans in the real world.

The first study, presented in Chapter \ref{chap:woz}, evaluated the interaction between the supervisor and the robot in a controlled and repeatable environment inspired by \gls{rat}, where the child was replace by a second robot simulating a child. The study showed that \gls{sparc} could allow a robot to learn, subsequently reducing the teacher's workload by decreasing the number of actions required from the teacher to control the robot.

Once \gls{sparc} demonstrated that it could allow a robot to learn and decreased the workload on the supervisor, we evaluated in Chapter \ref{chap:control} how the control provided to the teacher by \gls{sparc} could improve the learning process. To explore the impact of this teacher's control, we designed a second study comparing \gls{sparc} to \gls{irl} an alternative approach used to teach robots, but providing the teacher with less control over the robot's action. Results showed that the control provided to the teacher help them to guide the robot to relevant parts of the environment, thus improving the teaching process and making it safer.

Since the two first study demonstrated the efficiency of \gls{sparc} to teach a robot to interact in virtual environments and ensured that the control provided by \gls{sparc} could ensure the appropriateness of the robot's actions, \gls{sparc} was ready to be evaluated in a real human-robot interaction. We decided to select the context of tutoring children in the wild as it is a classic application of \gls{hri} and encompasses many challenges faced by robots interacting with humans. This last study applied \gls{sparc} to teach a robot a social and a technical policy to support child learning. This study had three main goals. First, demonstrating the applicability of \gls{sparc} in teaching robots to interact socially with humans, consequently addressing the thesis of this research. Secondly, exploring the impact during the teaching phase on the two humans involved in the triadic interaction. And finally, evaluating if, after having been taught by a human using \gls{sparc}, the robot could interact successfully with other humans in an autonomous manner. And results from this study demonstrated that during the teaching phase, the robot behaviour was efficient and appropriate to the interaction, but might require a relatively high workload from the teacher. And finally, when deployed autonomously, the robot could interact efficiently with children, leading to a similar policy than the teacher's and having similar effects on the children's behaviours.

%Difference contributions/summary??
\section{Contributions}\label{sec:conc_contribution}
\ES{From James}
This section will revisit the contributions outlined in the introduction (Chapter \ref{chap:intro}), with further expansion and explanation. 

The main scientific contributions of this thesis are as follows:
\begin{itemize}
	\item \textbf{Design of a new interaction framework for teaching agents in a safe way.} One of the main contribution of this research is \gls{sparc}: a new interaction framework enabling robot to learn to interact with humans from in situ human supervision. This method is corner stone of this thesis and aims to allow robots to be adaptive, while constantly ensuring an appropriate robot action policy and potentially requiring a decreasing workload from humans.
	
	\item \textbf{Evaluation of \gls{sparc} in three studies.} Throughout this research, \gls{sparc} have been tested in three study, which were developed to explores different aspects of the human-robot interactions involved in \gls{sparc} and culminating to teaching a robot to interact with human in the wild.
	
	\item \textbf{Demonstration of the importance of control over the robot's action when teaching a robot to interact.} As demonstrated by the last two studies, giving the teacher control over the robot's action has consequent impacts: it makes the teaching process safer, quicker and more efficient and it also give the opportunity to reduce the workload on the teacher.
	
	\item \textbf{Design of a lightweight algorithm to quickly learn from demonstration in complex environments.} Learning to interact in complex environments, where the states can be defined by a large number of dimensions is a challenging task, especially when the number of datapoints accessible to learn is low. By using a selective reduction of dimension decided by the teacher, this new algorithm allowed a robot to learn to interact in the complex environment that is child tutoring. 
	
	\item \textbf{Application of \gls{iml} to safely teach robots social autonomy from in situ human supervision.} Finally, the second main contribution of this research is a demonstration of online learning of interactive policy for \gls{hri} supported by human supervision. Similarly to agents using \gls{rl}, by using \gls{sparc} a robot learned an action policy by interacting with humans from scratch.
	
\ES{probably need to clean last bit and highlight why it matters}
\end{itemize}

%\subsection{Technical Contributions}
%In addition to the scientific contributions, this research project lead to software development for multiple projects.
%\begin{itemize}
%	\item Development of three different studies to evaluate \gls{sparc} in three different scenarios (supervised robot-robot interaction, virtual robot and real word \gls{hri}).
%	\item Partial development of a cognitive architecture and two tools for the \acrshort{dream} project (European FP7 project: 611391).
%	\item Development of an autonomous robot controller to support cardiac rehabilitation in the Human-Robot Interaction Strategies for Rehabilitation based on Socially Assistive Robotics project (Royal Academy of Engineering: IAPP\textbackslash1516\textbackslash137).
%	\item Development of a wizard interface for the Freeplay-Sandbox\footnote{\url{https://github.com/freeplay-sandbox}}. 
%\end{itemize}
%%%%%%%%%%%%%%%%%%%%%%%%%%%%%%%%%%%%%%%%%%%%%%%%%%%%%%%%%%%%%%%%%%%%%%%%%%
\section{Conclusion}\label{sec:conc_conc}


Providing a robot teacher with control can be a challenging task, but the rewards for doing so are consequent: the learning can be faster, more efficient and lightweight for the teacher. And most importantly, by preventing the robot's errors in early stage of the interaction, robots can be taught to interact in a wider range of situation where methods lacking control could not be applied.

\ES{from abstract}
Throughout evaluating \gls{sparc}, this work contributes to extend our knowledge on how humans teach autonomous agents and the impacts of features of the teacher-robot interface on the teaching process. It is found here that a supervised robot learning from a human could reduce the workload required to express a useful robot behaviour. Furthermore, providing the teacher with the opportunity to control the robot's behaviour improves substantially the teaching process.


reopening