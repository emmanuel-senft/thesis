\chapter{Contribution and Conclusion} \label{chap:conclusion}
\glsresetall

% TONY: Not needed, it's the conclusion, this is too obvious.

%This chapter first summarises the approach followed in this thesis and the main findings. Then, it presents and discuss briefly the contributions to the fields of social \gls{hri} and \gls{iml}. Finally, a conclusion outlines the primary outcomes of this work: the \acrfull{sparc}, its evaluation and applicability to \gls{hri} and beyond.

%%%%%%%%%%%%%%%%%%%%%%%%%%%%%%%%%%%%%%%%%%%%%%%%%%%%%%%%%%%%%%%%%%%%%%%%%%
\section{Summary}\label{sec:conc_summary}
%%%%%%%%%%%%%%%%%%%%%%%%%%%%%%%%%%%%%%%%%%%%%%%%%%%%%%%%%%%%%%%%%%%%%%%%%%

The main thesis defended in this work is that: 
\begin{quote}
	A robot can learn to interact meaningfully with people in an efficient and safe way by receiving supervision from a human teacher in control of the robot's behaviour. 
\end{quote}

To explore this thesis and the research questions arising from it, Chapter~\ref{chap:background}  reviewed the application field of social \gls{hri}. From this overview, we drew three requirements for a robot to interact efficiently in human environments. The robot needs (1) to constantly have an appropriate policy, (2) to be adaptive and (3) to impose a low workload on the person controlling and teaching the robot. Based on these requirements, we analysed the current controllers for robots in \gls{hri} and noted a need for a robot controller which met these requirements. However, we highlighted that, while not having been applied much to HRI yet, \gls{iml} does hold considerable promise for providing a robot with social interactive behaviour.

In Chapter~\ref{chap:sparc} we addressed this opportunity by proposing a new approach, the \gls{sparc}, to teach robots interactive capabilities and which would as such be applicable to \gls{hri}. \gls{sparc} aims to provide control over the robot's actions to a human and uses this supervision to learn a policy online. To achieve this goal, \gls{sparc} is based on a set of principles allowing this teaching interaction to be efficient: the teacher can select actions for the robot, the robot can propose actions to the teacher, the teacher can preempt robot's propositions before their automatic execution and, finally, the robot learns from the teacher's input and feedback to improve its policy. These principles ensure that the robot's executed policy is constantly appropriate, while reducing the human workload over time, leading to an autonomous behaviour if desired.

Once the method and the principles defining the interaction between the human teacher and the robot were set, we evaluated this approach in three studies with increasing ambition. Before testing \gls{sparc} in a real world interaction with people (in Chapter~\ref{chap:tutoring}), we needed to evaluate if the principles underlying \gls{sparc} could reduce the workload on the teacher and how the control over the robot's actions impacts the teaching process. A key effect to evaluate was if this control was sufficient to ensure a constant appropriate robot behaviour. The two first studies (Chapters~\ref{chap:woz} and~\ref{chap:control}) included people only as teachers, but not as the target of the teaching. This allowed us to gather initial information on \gls{sparc} in controlled and repeatable environments without having to tackle the challenges of interacting with people in the real world.

The first study, presented in Chapter~\ref{chap:woz}, evaluated the interaction between the supervisor and the robot in a controlled and repeatable environment inspired by \gls{rat}, where the patient was replaced by a second robot simulating a child. The study showed that \gls{sparc} could allow a robot to learn, subsequently reducing the teacher's workload by decreasing the number of actions required from the teacher to control the robot; and this decrease of human workload did not impact the performance in the interaction. In summary, by learning the robot could maintain a high performance in the interaction while requiring lower efforts from the human supervisor.

Once \gls{sparc} was demonstrated to allow a robot to learn and decrease the workload on its supervisor; we evaluated in Chapter~\ref{chap:control} how the control provided by \gls{sparc} to the teacher could improve the learning process. To explore this aspect, we designed a second study comparing \gls{sparc} to \gls{irl}, an alternative approach used to teach robots, but providing the teacher with little control over the robot's action. The results showed that the additional control provided to the teacher helped them guide the robot to relevant parts of the environment, thus improving the teaching process and making it safer and easier.

Since the two first studies demonstrated the efficiency of \gls{sparc} to teach a robot to interact in virtual environments and ensured that the control provided by \gls{sparc} could guarantee the appropriateness of the robot's actions, \gls{sparc} was ready to be evaluated in a real human-robot interaction scenario. We decided to select the context of tutoring children in the wild as this classic application of \gls{hri} encompasses many challenges faced by robots interacting with humans. This last study applied \gls{sparc} to teach a robot a social and a technical policy to support child learning. This study had three main goals. First, demonstrating the applicability of \gls{sparc} in teaching robots to interact socially with humans, consequently addressing the main thesis of this research. Secondly, exploring the impact of \gls{sparc} during the teaching phase on the two people involved in the triadic interaction. And finally, evaluating if, after having been taught by a person using \gls{sparc}, the robot could interact successfully with other people in an autonomous manner. Results from this study demonstrated that during the teaching phase, the robot behaviour was efficient and appropriate to the interaction, but might have required a relatively high workload from the teacher. And finally, when deployed autonomously, the robot could interact efficiently with children, leading to a policy selecting actions similar to those selected by the teacher and having similar effects on the children's behaviours. Consequently, \gls{sparc} was shown to enable the teaching of complex social behaviours in a multimodal and multidimensional environment where incorrect actions have important consequences.

%Difference contributions/summary??
\section{Contributions}\label{sec:conc_contribution}
%\ES{From James}
%This section will revisit the contributions outlined in the introduction (Chapter~\ref{chap:intro}), with further expansion and explanation. 
As mentioned in the introduction, the main scientific contributions of this thesis are as follows:
\begin{itemize}
	\item \textbf{Design of a new interaction framework for teaching agents in a safe manner.} One of the main contribution of this research is \gls{sparc}: a new interaction framework enabling robots to learn to interact with people from \textit{in situ} human supervision. This method is the cornerstone of this thesis and aims to allow robots to be adaptive, while constantly ensuring an appropriate robot policy and requiring a low workload from humans.
	
	\item \textbf{Evaluation of \gls{sparc} in three studies.} \gls{sparc} has been tested in three studies, which were specifically developed to explore different aspects of the human-robot interaction involved in \gls{sparc} and culminated with the teaching of a robot to interact with humans in the wild.
	
	\item \textbf{Demonstration of the importance of control over the robot's action when teaching a robot to interact.} As demonstrated by the second study, giving the teacher control over the robot's actions has consequent impacts: it makes the teaching process safer, quicker and more efficient and it also provides the opportunity to reduce the workload on the teacher.
	
	\item \textbf{Design of a lightweight algorithm to quickly learn from demonstration in complex environments.} Learning to interact in complex environments, where states and actions are defined on a large number of dimensions is a challenging task, especially when the number of datapoints accessible to learn is low. By using a selective reduction of dimensions decided by the teacher, this new algorithm inspired by nearest neighbours classifation allowed a robot to learn to interact in the complex environment, i.e. child tutoring with few datapoints. 
	
	\item \textbf{Application of \gls{iml} to safely teach robots social autonomy from \textit{in situ} human supervision.} Finally, the second main contribution of this research is a demonstration of online learning of interactive policy for \gls{hri} supported by human supervision. Through \gls{sparc} a robot learned a policy from scratch by interacting with people and applied it successfully in further autonomous interactions. The complexity of the world in which the robot interacted should be noted: the robot needed both technical and social knowledge about the interaction to behave properly. Furthermore, both the action and state spaces were large and multimodal. However, only a limited number of interactions were required to reach an efficient policy.	This has real world implications as it shows that by using \gls{sparc}, an agent can learn a effective social policy from a generic perception of the world and a large set of actions. 
	
\end{itemize}

%\subsection{Technical Contributions}
%In addition to the scientific contributions, this research project lead to software development for multiple projects.
%\begin{itemize}
%	\item Development of three different studies to evaluate \gls{sparc} in three different scenarios (supervised robot-robot interaction, virtual robot and real word \gls{hri}).
%	\item Partial development of a cognitive architecture and two tools for the \acrshort{dream} project (European FP7 project: 611391).
%	\item Development of an autonomous robot controller to support cardiac rehabilitation in the Human-Robot Interaction Strategies for Rehabilitation based on Socially Assistive Robotics project (Royal Academy of Engineering: IAPP\textbackslash1516\textbackslash137).
%	\item Development of a wizard interface for the Freeplay-Sandbox\footnote{\url{https://github.com/freeplay-sandbox}}. 
%\end{itemize}
%%%%%%%%%%%%%%%%%%%%%%%%%%%%%%%%%%%%%%%%%%%%%%%%%%%%%%%%%%%%%%%%%%%%%%%%%%
\section{Conclusion}\label{sec:conc_conc}

The thesis presented here is that a robot can learn to interact meaningfully with humans in an efficient and safe way by receiving supervision from a human teacher in control of the robot's behaviour. 
%
%Using humans to teach robot provide significant advantages compared to having robots learning on their own. Furthermore, while being a challenging task, providing a robot teacher with control over the robot's actions improve these advantages: the learning can be faster, more efficient and lightweight for the teacher. And most importantly, by preventing the robot's errors in early stage of the interaction, robots can be taught to interact in a wider range of situation where methods lacking control could not be applied.
%
To support this thesis, this research work proposed \acrfull{sparc}, a new machine learning framework for artificial agents seeking to give to the teacher full control over the agent's actions. The agent learns from this supervision in a safe and efficient way and progressively becomes autonomous. By proposing and evaluating \gls{sparc} in three studies, this work contributes to increase our knowledge in both \gls{hri} and \gls{iml}, especially on how robots could be taught to interact in complex and sensitive environments, such as therapy or educational use scenarios. It has been shown that, while being a challenging task, providing a robot teacher with control over the robot's actions leads to significant advantages: the learning can be fast, efficient and lightweight for the teacher. And most importantly, by preventing the robot's errors in early stage of the interaction, robots can be taught to interact in high-stakes environments, where methods lacking this sort of control cannot be applied.

Finally, by exhibiting its application in child tutoring, \gls{sparc} confirms its applicability to complex and sensitive environments such as \gls{hri}. This demonstrates the potential for \gls{sparc} to provide non-experts in computing the ability to teach robots rich behaviours in large and complex environments and shows the potential to foster the deployment of robots in more interactive domains.